\begin{abstract}

It is understood by the community that enhancing methods for exhaustively verifying such transformations allows for a more wide\-spread adoption of model-driven engineering in industry. A variety of proposals for the verification of transformations have arisen in the past few years. However, as can be seen for verification of ATL transformations, the majority of transformation verification techniques are either based on non-exhaustive testing or on proof methods that require human assistance and/or are not complete.

In this paper, we describe our tool for statically verifying model transformations written in the DSLTrans model transformation language.  As DSLTrans is Turing-incomplete, this reduction in expressivity allows us to use a symbolic-execution approach to generate representations of all possible input models to the transformation. We then verify pre-/post-condition contracts on these representations, which in turn verifies the transformation itself.

We describe the architecture of our tool, which connects to JetBrains' MPS in order to provide a graphical or textual editor with auto-complete to create both the transformations and the contracts.  Our tool is structured to work with models as the primary artifacts to be manipulated, such that modularity is provided and the number of bugs is reduced. In particular, our technique is based on graph-matching, using the graph matching/rewriting library T-Core. DSLTrans is also being integrated as the model-to-model transformation language within MPS, allowing all users of this tool immediate access to transformation verification capabilities.

The efficacy of our tool is demonstrated by showcasing the building of an industrial transformation within MPS, and then the verification of a number of contracts. This mbeddr transformation has the intent of translating high-level code into low-level C, to reduce the burden on the programmer and reduce mistakes. Our prover will demonstrate that a number of statements are correctly translated, despite the significant size of the transformation.

\end{abstract}