We have some evidence that SyVOLT scales to transformations of practical
interest. This has been empirically shown by applying it to DSLTrans
transformations up to over 60 rules, and ATL transformations up to 13
rules~\cite{Oakes2016}. From our own experience with DSLTrans, the size of a
DSLTrans transformations varies widely, with the average size ranging from 10 up
to 50 rules. The average size of an ATL transformation is around 20 rules. [cite
Manuel's paper, should we keep this?].\markus{Well, if we look at the case study
in this paper (mbeddr), then we can definitely see that to really verify mbeddr,
we'll have waaay more rules.} Even though our technique is exhaustive, our
approach takes relatively short amounts of time to prove contracts. For example,
our experiments with industrial transformations~\cite{Oakes2016} show that contracts
can be verified within a few minutes. In Gehan Selim's PhD
thesis~\cite{Selim2015} further evidence of SyVOLT's performance is given when
verifying a relatively large model transformation for giving semantics to the
UML-RT language in terms of the Kiltera process language~\cite{PosseDingel2014}.
SyVOLT's symbolic execution engine is fully
homegrown~\cite{LucioVang}\markus{Does ``homegrown'' have positive connotations?
Not sure.} and does not depend on third-party solvers. Although this has implied
a large effort to build the codebase, it has also allowed us to have the
required control over the code to iteratively optimize the engine for both space
and time economy.
\cite{Selim2014} demonstrates that our prover is substantially faster than
similar approaches based on SAT solvers.