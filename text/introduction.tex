\subsection{Model Transformations and Verification}

Model-Driven Development (MDD) is a software engineering approach that uses models - typically represented as graphs - as first class citizens to create and evolve software systems~\cite{Hailpern:2006vd}. This approach is becoming increasingly common in industrial applications for the development of complex software due to the modularity of this approach~\cite{daghsen:hal-00660252,Giese2010}.


Model transformations, written in model transformation languages, have become the main means
for manipulating models in model-driven engineering~\cite{Brambilla12}. These transformations are an excellent compromise between strong theoretical foundations and applicability to real-world problems~\cite{LucioSoSyM2015}. In particular, model transformations allow for mathematical treatment based on foundations of graphs and graph transformations, and can natively
manipulate domain-specific concepts expressed in metamodels.


% What is transformation verification?

\bentley{Should be rewritten, as it was copied from SOSYM paper}

Due to the importance of MDD in both the academic and the industrial arenas,
the verification of model transformations is of prime importance. This is because
the correctness of software built using model-driven engineering techniques
typically relies on the correctness of operations executed using model
transformations. As well, there is a strong demand for tools that allow the building of verified
software, especially in industries where quality and safety standards have to be met.


\subsection{SyVOLT Contract Verification Tool}

In this paper we address this issue by detailing our technique for verifying visual pre-/post-condition contracts on DSLTrans model transformations. A contract is said to hold on a transformation if it holds on the transformation's input-output pairs. That is, for all input models where the contract's pre-condition
holds, the contract's post-condition also holds in the corresponding output model produced by executing the transformation. Traceability constraints between the elements of the input and output models may also be required. Otherwise, the contract does not hold, and consequently, the transformation does not correctly implement the contract.

For example, this paper considers as an example an extended version of the well-known \FTP
transformation from the ATL zoo~\cite{atlZoo}, where mother(s), father(s),
daughter(s) and son(s) belonging to a family are translated into men and women
who are members of a community. One possible contract would try to assert that,
for any input model containing a family that includes a mother and a daughter,
a man is produced in the output community. We would expect the contract not to
hold for the \FTP transformation, because there can exist families that are composed of
only a mother and her daughter.

The main contribution of our technique is that, if
our prover demonstrates that the contract holds, then it will hold for any input
model given to the DSLTrans model transformation. We can thus guarantee the
user can safely execute the model transformation without any need for additional
testing or runtime checking, as seen in other verification approaches (cf. Section~\ref{sec:related}). Our contract language is based on pre-/post-condition contracts, but also includes propositional logic operators
for combining contracts, as further discussed in~\cite{Oakes2016}.


We prove that contracts hold or not on transformations defined in a model transformation language called
DSLTrans~\cite{Barroca2011}. A theoretical framework has been developed for
the DSLTrans model transformation language in which pre-/post-condition
contracts can be shown to hold for all those input/output model pairs resulting from executing a given
DSLTrans model transformation, or to not hold for at least one of those
input/output pairs~\cite{Lucio2014}. A fully automatic property prover based on
this theory has been shown to be applicable to industrial
problems~\cite{Selim2014}, as well as the declarative subset of ATL transformations~\cite{Oakes2016}.

\subsection{Paper Structure}

This paper will therefore discuss our tool for verifying structural contracts on sizable transformations. First, we begin by describing DSLTrans and an introductory example in Section~\ref{sec:background}. Section~\ref{sec:contract_proving} provides an overview of our contract proving technique, while Section~\ref{sec:prover_tool} describes our tool SyVOLT along with some of the required technologies.

In Section~\ref{sec:mbeddr_transformation}, we begin our examination of our industrial case study, the mbeddr transformation. This examination continues in Section~\ref{sec:mbeddr_contracts}, where we examine contracts to be proved on this transformation, along with their semantic meaning. Section~\ref{sec:results} presents the results and performance metrics of contract proof on this case study.

A discussion of our results is presented in Section~\ref{sec:discussion}, while a comparison to other model transformation verification approaches is found in Section~\ref{sec:related}. Finally, we conclude in Section~\ref{sec:conclusion}.